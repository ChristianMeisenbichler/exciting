%                **** IMPORTANT NOTICE *****
% This LaTeX file has been automatically produced by ProTeX v. 1.1
% Any changes made to this file will likely be lost next time
% this file is regenerated from its source. Send questions 
% to Arlindo da Silva, dasilva@gsfc.nasa.gov
 
%------------------------ PREAMBLE --------------------------
\documentclass[11pt]{article}
\usepackage{amsmath}
\usepackage{epsfig}
\usepackage{hangcaption}
\usepackage{tabularx}
\textheight     9in
\topmargin      0pt
\headsep        1cm
\headheight     0pt
\textwidth      6in
\oddsidemargin  0in
\evensidemargin 0in
\marginparpush  0pt
\pagestyle{myheadings}
\markboth{}{}
%-------------------------------------------------------------
\setlength{\parskip}{0pt}
\setlength{\parindent}{0pt}
\setlength{\baselineskip}{11pt}
 
%--------------------- SHORT-HAND MACROS ----------------------
\def\bv{\begin{verbatim}}
\def\ev{\end{verbatim}}
\def\be{\begin{equation}}
\def\ee{\end{equation}}
\def\bea{\begin{eqnarray}}
\def\eea{\end{eqnarray}}
\def\bi{\begin{itemize}}
\def\ei{\end{itemize}}
\def\bn{\begin{enumerate}}
\def\en{\end{enumerate}}
\def\bd{\begin{description}}
\def\ed{\end{description}}
\def\({\left (}
\def\){\right )}
\def\[{\left [}
\def\]{\right ]}
\def\<{\left  \langle}
\def\>{\right \rangle}
\def\cI{{\cal I}}
\def\diag{\mathop{\rm diag}}
\def\tr{\mathop{\rm tr}}
%-------------------------------------------------------------

\markboth{Left}{Source File: xsmain.F90,  Date: Fri May 23 14:38:01 CEST 2008
}

 
\title{The XS/EXCITING Code (eXited States) Manual \\ Version 0.9}
\author{{\sc S. Sagmeister and C. Ambrosch-Draxl}\\ {\em }}
\date{}
\begin{document}
\maketitle
\tableofcontents
\newpage
 
%..............................................
\section{Introduction}

     Welcome to the {\sf XS/EXCITING} code developers' manual
     This is supposed to collect the routines and modules belonging
     only to the excited states (TDDFT and BSE) part into one document.
     \\\\
     S. Sagmeister\\
     Leoben, 2008
   
%/////////////////////////////////////////////////////////////
\newpage

\markboth{Left}{Source File: bandgap.F90,  Date: Fri May 23 14:38:01 CEST 2008
}

 
%/////////////////////////////////////////////////////////////
\section{Routine/Function Prologues} \label{app:ProLogues}
\subsection{bandgapgrid (Source File: bandgap.F90)}


\bigskip{\sf INTERFACE:}
\begin{verbatim}   subroutine bandgap(n,e,ef,egf,ego,ikgf,ikgo,istho)\end{verbatim}{\em USES:}
\begin{verbatim}     use modmain\end{verbatim}
{\sf DESCRIPTION:\\ }


     Determines the fundamental and optical band gap if present.
  
\bigskip{\sf REVISION HISTORY:}
\begin{verbatim}     Created July 2007 (Sagmeister)\end{verbatim}

\markboth{Left}{Source File: bse.F90,  Date: Fri May 23 14:38:01 CEST 2008
}


\markboth{Left}{Source File: chi0upd.F90,  Date: Fri May 23 14:38:01 CEST 2008
}


\markboth{Left}{Source File: connecta.F90,  Date: Fri May 23 14:38:01 CEST 2008
}

 
%/////////////////////////////////////////////////////////////
 
\mbox{}\hrulefill\ 
 
\subsection{connecta (Source File: connecta.F90)}


\bigskip{\sf INTERFACE:}
\begin{verbatim} subroutine connecta(cvec,nv,np,vvl,vpl,dv,dp)\end{verbatim}{\em INPUT/OUTPUT PARAMETERS:}
\begin{verbatim}     cvec : matrix of (reciprocal) lattice vectors stored column-wise
           (in,real(3,3))
     nv   : number of vertices (in,integer)
     np   : number of connecting points (in,integer)
     vvl  : vertex vectors in lattice coordinates (in,real(3,nv))
     vpl  : connecting point vectors in lattice coordinates (out,real(3,np))
     dv   : cummulative distance to each vertex (out,real(nv))
     dp   : cummulative distance to each connecting point (out,real(np))\end{verbatim}
{\sf DESCRIPTION:\\ }


     Generates a set of points which interpolate between a given set of vertices.
     Vertex points are supplied in lattice coordinates in the array {\tt vvl} and
     converted to Cartesian coordinates with the matrix {\tt cvec}. Interpolating
     points are stored in the array {\tt vpl}. The cummulative distances to the
     vertices and points along the path are stored in arrays {\tt dv} and
     {\tt dp}, respectively. Based upon the routine {\tt connect}.
  
\bigskip{\sf REVISION HISTORY:}
\begin{verbatim}     Created June 2003 (JKD)
     Modifications 2007 (Sagmeister)\end{verbatim}

\markboth{Left}{Source File: ctdfrac.F90,  Date: Fri May 23 14:38:01 CEST 2008
}


\markboth{Left}{Source File: df.F90,  Date: Fri May 23 14:38:01 CEST 2008
}


\markboth{Left}{Source File: dfgather.F90,  Date: Fri May 23 14:38:01 CEST 2008
}


\markboth{Left}{Source File: dfq.F90,  Date: Fri May 23 14:38:01 CEST 2008
}


\markboth{Left}{Source File: dfqoscbo.F90,  Date: Fri May 23 14:38:01 CEST 2008
}


\markboth{Left}{Source File: dfqoschd.F90,  Date: Fri May 23 14:38:01 CEST 2008
}


\markboth{Left}{Source File: dfqoscwg.F90,  Date: Fri May 23 14:38:01 CEST 2008
}


\markboth{Left}{Source File: dftim.F90,  Date: Fri May 23 14:38:01 CEST 2008
}


\markboth{Left}{Source File: dyson.F90,  Date: Fri May 23 14:38:01 CEST 2008
}


\markboth{Left}{Source File: ematbdcmbs.F90,  Date: Fri May 23 14:38:01 CEST 2008
}


\markboth{Left}{Source File: ematbdlims.F90,  Date: Fri May 23 14:38:01 CEST 2008
}


\markboth{Left}{Source File: ematgather.F90,  Date: Fri May 23 14:38:01 CEST 2008
}


\markboth{Left}{Source File: ematgntsum.F90,  Date: Fri May 23 14:38:01 CEST 2008
}


\markboth{Left}{Source File: ematqalloc.F90,  Date: Fri May 23 14:38:01 CEST 2008
}


\markboth{Left}{Source File: ematqdealloc.F90,  Date: Fri May 23 14:38:01 CEST 2008
}


\markboth{Left}{Source File: ematq.F90,  Date: Fri May 23 14:38:01 CEST 2008
}


\markboth{Left}{Source File: ematqk1.F90,  Date: Fri May 23 14:38:01 CEST 2008
}


\markboth{Left}{Source File: ematqk.F90,  Date: Fri May 23 14:38:01 CEST 2008
}


\markboth{Left}{Source File: ematqkgir.F90,  Date: Fri May 23 14:38:01 CEST 2008
}


\markboth{Left}{Source File: ematqkgmt.F90,  Date: Fri May 23 14:38:01 CEST 2008
}


\markboth{Left}{Source File: ematrad.F90,  Date: Fri May 23 14:38:01 CEST 2008
}


\markboth{Left}{Source File: emattest.F90,  Date: Fri May 23 14:38:01 CEST 2008
}


\markboth{Left}{Source File: emattim.F90,  Date: Fri May 23 14:38:01 CEST 2008
}


\markboth{Left}{Source File: epsconv.F90,  Date: Fri May 23 14:38:01 CEST 2008
}


\markboth{Left}{Source File: exccoulint.F90,  Date: Fri May 23 14:38:01 CEST 2008
}


\markboth{Left}{Source File: filedel.F90,  Date: Fri May 23 14:38:01 CEST 2008
}


\markboth{Left}{Source File: findexciton.F90,  Date: Fri May 23 14:38:01 CEST 2008
}


\markboth{Left}{Source File: findgntn0.F90,  Date: Fri May 23 14:38:01 CEST 2008
}


\markboth{Left}{Source File: findgroupq.F90,  Date: Fri May 23 14:38:01 CEST 2008
}


\markboth{Left}{Source File: findkmapkq.F90,  Date: Fri May 23 14:38:01 CEST 2008
}


\markboth{Left}{Source File: findocclims.F90,  Date: Fri May 23 14:38:01 CEST 2008
}


\markboth{Left}{Source File: findsymi.F90,  Date: Fri May 23 14:38:01 CEST 2008
}

 
%/////////////////////////////////////////////////////////////
 
\mbox{}\hrulefill\ 
 
\subsection{findsymi (Source File: findsymi.F90)}


\bigskip{\sf INTERFACE:}
\begin{verbatim} subroutine findsymi(epslat,maxsymcrys,nsymcrys,symlat,lsplsymc,vtlsymc,isymlat,&
      scimap)\end{verbatim}
{\sf DESCRIPTION:\\ }


     Throughout the code the symmetries are understood to be applied in a way 
     $$ (\alpha_S|\alpha_R|{\bf t}) {\bf x} = \alpha_S\alpha_R
     ({\bf x}+{\bf t})$$
     which is different from the commonly used definition
     $\{\alpha|\tau\}x=\alpha x+\tau$ -- see routine {\tt findsymcrys}.
     This difference affects the inverse of the fractional translation
     but has no effect on the inverse of the rotational part, so the inverse
     spacegroup symmetry operations are the same for both definitions.
  
\bigskip{\sf REVISION HISTORY:}
\begin{verbatim}     Created April 2007 (Sagmeister)\end{verbatim}

\markboth{Left}{Source File: ftfun.F90,  Date: Fri May 23 14:38:01 CEST 2008
}


\markboth{Left}{Source File: fxc\_alda\_check.F90,  Date: Fri May 23 14:38:01 CEST 2008
}


\markboth{Left}{Source File: fxc\_bse\_ma03.F90,  Date: Fri May 23 14:38:01 CEST 2008
}


\markboth{Left}{Source File: fxc\_lrcd.F90,  Date: Fri May 23 14:38:01 CEST 2008
}


\markboth{Left}{Source File: fxc\_lrc.F90,  Date: Fri May 23 14:38:01 CEST 2008
}


\markboth{Left}{Source File: fxc\_pwca.F90,  Date: Fri May 23 14:38:01 CEST 2008
}

 
%/////////////////////////////////////////////////////////////
 
\mbox{}\hrulefill\ 
 
\subsection{xcd\_pwca (Source File: fxc\_pwca.F90)}


\bigskip{\sf INTERFACE:}
\begin{verbatim} subroutine xcd_pwca(n,rho,dvx,dvc)\end{verbatim}{\em INPUT/OUTPUT PARAMETERS:}
\begin{verbatim}     n     : number of density points (in,integer)
     rho   : charge density (in,real(n))
     dvx   : exchange potential derivative (out,real(n))
     dvc   : correlation potential derivative (out,real(n))\end{verbatim}
{\sf DESCRIPTION:\\ }


     Spin-unpolarised exchange-correlation potential derivative of the 
     Perdew-Wang
     parameterisation of the Ceperley-Alder electron gas,
     Phys. Rev. B 45, 13244
     (1992) and Phys. Rev. Lett. 45, 566 (1980). Based upon the routine
     {\tt xc\_pwca}.
  
\bigskip{\sf REVISION HISTORY:}
\begin{verbatim}     Created February 2007 (Sagmeister)\end{verbatim}

\markboth{Left}{Source File: genapwcmt.F90,  Date: Fri May 23 14:38:01 CEST 2008
}


\markboth{Left}{Source File: genfilextread.F90,  Date: Fri May 23 14:38:01 CEST 2008
}


\markboth{Left}{Source File: genfilname.F90,  Date: Fri May 23 14:38:01 CEST 2008
}


\markboth{Left}{Source File: gengqvec.F90,  Date: Fri May 23 14:38:01 CEST 2008
}

 
%/////////////////////////////////////////////////////////////
 
\mbox{}\hrulefill\ 
 
\subsection{gengqvec (Source File: gengqvec.F90)}


\bigskip{\sf INTERFACE:}
\begin{verbatim} subroutine gengqvec(iq,vpl,vpc,ngp,igpig,vgpl,vgpc,gpc,tpgpc)\end{verbatim}{\em USES:}
\begin{verbatim}   use modmain
   use modxs\end{verbatim}{\em INPUT/OUTPUT PARAMETERS:}
\begin{verbatim}     vpl   : p-point vector in lattice coordinates (in,real(3))
     vpc   : p-point vector in Cartesian coordinates (in,real(3))
     ngp   : number of G+p-vectors returned (out,integer)
     igpig : index from G+p-vectors to G-vectors (out,integer(ngkmax))
     vgpl  : G+p-vectors in lattice coordinates (out,real(3,ngkmax))
     vgpc  : G+p-vectors in Cartesian coordinates (out,real(3,ngkmax))
     gpc   : length of G+p-vectors (out,real(ngkmax))
     tpgpc : (theta, phi) coordinates of G+p-vectors (out,real(2,ngkmax))\end{verbatim}
{\sf DESCRIPTION:\\ }


     Generates a set of ${\bf G+p}$-vectors for the input ${\bf p}$-point with
     length less than {\tt gkmax}. These are used as the plane waves in the APW
     functions. Also computes the spherical coordinates of each vector.
     Based on gengpvec.
  
\bigskip{\sf REVISION HISTORY:}
\begin{verbatim}     Created October 2006 (Sagmeister)\end{verbatim}

\markboth{Left}{Source File: genlocmt.F90,  Date: Fri May 23 14:38:01 CEST 2008
}


\markboth{Left}{Source File: genloss.F90,  Date: Fri May 23 14:38:01 CEST 2008
}


\markboth{Left}{Source File: genparidxran.F90,  Date: Fri May 23 14:38:01 CEST 2008
}


\markboth{Left}{Source File: genpmat2.F90,  Date: Fri May 23 14:38:01 CEST 2008
}

 
%/////////////////////////////////////////////////////////////
 
\mbox{}\hrulefill\ 
 
\subsection{genpmat2 (Source File: genpmat2.F90)}


\bigskip{\sf INTERFACE:}
\begin{verbatim} subroutine genpmat2(ngp,igpig,vgpc,ripaa,ripalo,riploa,riplolo,apwcmt,locmt, &
      evecfv,evecsv,pmat)\end{verbatim}{\em USES:}
\begin{verbatim}   use modmain\end{verbatim}{\em INPUT/OUTPUT PARAMETERS:}
\begin{verbatim}     ngp    : number of G+p-vectors (in,integer)
     igpig  : index from G+p-vectors to G-vectors (in,integer(ngkmax))
     vgpc   : G+p-vectors in Cartesian coordinates (in,real(3,ngkmax))
     ripaa  : gradient of radial functions time spherical harmonics (APW-APW)
              (in,complex(apwordmax,lmmaxapw,apwordmax,lmmaxapw,natmtot,3))
     ripalo : gradient of radial functions time spherical harmonics (APW-lo)
              (in,complex(apwordmax,lmmaxapw,nlomax,-lolmax:lolmax,natmtot,3))
     riploa : gradient of radial functions time spherical harmonics (lo-APW)
              (in,complex(nlomax,-lolmax:lolmax,apwordmax,lmmaxapw,natmtot,3))
     riplolo: gradient of radial functions time spherical harmonics (lo-lo)
              (in,complex(nlomax,-lolmax:lolmax,nlomax,-lolmax:lolmax,natmtot,3))
     apwcmt : APW expansion coefficients coefficients
              (in,complex(nstfv,apwordmax,lmmaxapw,natmtot))
     locmt  : local orbitals expansion coefficients coefficients
              (in,complex(nstfv,nlomax,-lolmax:lolmax,natmtot))
     evecfv : first-variational eigenvector (in,complex(nmatmax,nstfv))
     evecsv : second-variational eigenvectors (in,complex(nstsv,nstsv))
     pmat   : momentum matrix elements (out,complex(3,nstsv,nstsv))\end{verbatim}
{\sf DESCRIPTION:\\ }


     Calculates the momentum matrix elements
     $$ p_{ij}=\langle\Psi_{i,{\bf k}}|-i\nabla|\Psi_{j,{\bf k}}\rangle. $$
     The gradient is applied explicitly only to the radial functions and 
     corresponding spherical harmonics for the muffin-tin part. In the 
     interstitial region the gradient is evaluated analytically.
     Parts taken from the routine {\tt genpmat}.
  
\bigskip{\sf REVISION HISTORY:}
\begin{verbatim}     Created April 2008 (Sagmeister)\end{verbatim}

\markboth{Left}{Source File: genpwmat.F90,  Date: Fri May 23 14:38:01 CEST 2008
}

 
%/////////////////////////////////////////////////////////////
 
\mbox{}\hrulefill\ 
 
\subsection{genpwmat (Source File: genpwmat.F90)}


\bigskip{\sf INTERFACE:}
\begin{verbatim} subroutine genpwmat(vpl,ngpmax,ngp,vgpc,gpc,igpig,ylmgp,sfacgp,vklk,ngkk, &
      igkigk,apwalmk,evecfvk,evecsvk,vklkp,ngkkp,igkigkp,apwalmkp,evecfvkp, &
      evecsvkp,pwmat)\end{verbatim}{\em USES:}
\begin{verbatim}   use modmain
   use modxs\end{verbatim}{\em INPUT/OUTPUT PARAMETERS:}
\begin{verbatim} \end{verbatim}
{\sf DESCRIPTION:\\ }

    Calculates the matrix elements of the plane wave
     $$ p_{ij}=\langle\Psi_{i,{\bf k}}|e^{-i({\bf G}+{\bf q}){\bf r}}|
        \Psi_{j,{\bf k}}\rangle. $$
    Straightforward implementation for checking.
  
\bigskip{\sf REVISION HISTORY:}
\begin{verbatim}     Created November 2007 (Sagmeister)\end{verbatim}

\markboth{Left}{Source File: genqvkloff.F90,  Date: Fri May 23 14:38:01 CEST 2008
}


\markboth{Left}{Source File: gensigma.F90,  Date: Fri May 23 14:38:01 CEST 2008
}


\markboth{Left}{Source File: genstar.F90,  Date: Fri May 23 14:38:01 CEST 2008
}

 
%/////////////////////////////////////////////////////////////
 
\mbox{}\hrulefill\ 
 
\subsection{genstar (Source File: genstar.F90)}


\bigskip{\sf INTERFACE:}
\begin{verbatim} subroutine genstar(tsmgrq,ngridp,nppt,npptnr,vpl,vplnr,ivpnr,ipmap,ipmapnr, &
      nsymcrysst,scmapst,ipstmapipnr,stmap,stmapsymc)\end{verbatim}{\em USES:}
\begin{verbatim}   use modmain
   use modxs\end{verbatim}
{\sf DESCRIPTION:\\ }


     Generates the stars for the $k$-point set as reference to crystal
     symmetries. For a non-zero q-point the little group of q is taken
     instead of the full symmetry group.
  
\bigskip{\sf REVISION HISTORY:}
\begin{verbatim}     Created December 2007 (Sagmeister)\end{verbatim}

\markboth{Left}{Source File: genstark.F90,  Date: Fri May 23 14:38:01 CEST 2008
}

 
%/////////////////////////////////////////////////////////////
 
\mbox{}\hrulefill\ 
 
\subsection{genstark (Source File: genstark.F90)}


\bigskip{\sf INTERFACE:}
\begin{verbatim} subroutine genstark\end{verbatim}{\em USES:}
\begin{verbatim}   use modmain
   use modxs\end{verbatim}
{\sf DESCRIPTION:\\ }


     Generates the stars for the ${\bf k}$-point set as reference to crystal
     symmetries. For a non-zero ${\bf q}$-point the little group of ${\bf q}$
     is taken instead of the full symmetry group.
  
\bigskip{\sf REVISION HISTORY:}
\begin{verbatim}     Created December 2007 (Sagmeister)\end{verbatim}

\markboth{Left}{Source File: gensumrls.F90,  Date: Fri May 23 14:38:01 CEST 2008
}


\markboth{Left}{Source File: gensymdf.F90,  Date: Fri May 23 14:38:01 CEST 2008
}


\markboth{Left}{Source File: gentetlink.F90,  Date: Fri May 23 14:38:01 CEST 2008
}

 
%/////////////////////////////////////////////////////////////
 
\mbox{}\hrulefill\ 
 
\subsection{gentetlink (Source File: gentetlink.F90)}


\bigskip{\sf INTERFACE:}
\begin{verbatim} subroutine gentetlink(vpl)\end{verbatim}{\em USES:}
\begin{verbatim}   use modmain
   use modtetra\end{verbatim}
{\sf DESCRIPTION:\\ }


     Generates an array connecting the tetrahedra of the $\mathbf{k}$-point with
     the ones of the  $\mathbf{k}+\mathbf{q}$-point. Interface routine
     referencing the {\tt libbzint} library of Ricardo Gomez-Abal.
  
\bigskip{\sf REVISION HISTORY:}
\begin{verbatim}     Created January 2008 (SAG)\end{verbatim}

\markboth{Left}{Source File: gentim.F90,  Date: Fri May 23 14:38:01 CEST 2008
}


\markboth{Left}{Source File: genwgrid.F90,  Date: Fri May 23 14:38:01 CEST 2008
}


\markboth{Left}{Source File: genwiq2xs.F90,  Date: Fri May 23 14:38:01 CEST 2008
}

 
%/////////////////////////////////////////////////////////////
 
\mbox{}\hrulefill\ 
 
\subsection{genwiq2xs (Source File: genwiq2xs.F90)}


\bigskip{\sf INTERFACE:}
\begin{verbatim} subroutine genwiq2xs(flag,iq,igq1,igq2,clwt)\end{verbatim}{\em USES:}
\begin{verbatim}   use modmain
   use modxs
   use m_genfilname
   use m_getunit\end{verbatim}
{\sf DESCRIPTION:\\ }


     Effective integrals of Coulomb interaction.
  
\bigskip{\sf REVISION HISTORY:}
\begin{verbatim}     Created February 2008 (SAG)\end{verbatim}

\markboth{Left}{Source File: genylmgq.F90,  Date: Fri May 23 14:38:01 CEST 2008
}

 
%/////////////////////////////////////////////////////////////
 
\mbox{}\hrulefill\ 
 
\subsection{genylmgq (Source File: genylmgq.F90)}


\bigskip{\sf INTERFACE:}
\begin{verbatim} subroutine genylmgq(iq,lmax)\end{verbatim}{\em USES:}
\begin{verbatim} use modmain
 use modxs\end{verbatim}
{\sf DESCRIPTION:\\ }


     Generates a set of spherical harmonics, $Y_{lm}(\widehat{{\bf G}+{\bf q}})$,
     with angular
     momenta up to {\tt lmax} for the set of ${\bf G+q}$-vectors. Based upon
     the routine genylmg.
  
\bigskip{\sf REVISION HISTORY:}
\begin{verbatim}     Created October 2006 (Sagmeister)\end{verbatim}

\markboth{Left}{Source File: getapwdlm.F90,  Date: Fri May 23 14:38:01 CEST 2008
}


\markboth{Left}{Source File: getdevaldoccsv.F90,  Date: Fri May 23 14:38:01 CEST 2008
}


\markboth{Left}{Source File: getemat.F90,  Date: Fri May 23 14:38:01 CEST 2008
}


\markboth{Left}{Source File: getevalsv0.F90,  Date: Fri May 23 14:38:01 CEST 2008
}


\markboth{Left}{Source File: getevecfv0.F90,  Date: Fri May 23 14:38:01 CEST 2008
}


\markboth{Left}{Source File: getngqmax.F90,  Date: Fri May 23 14:38:01 CEST 2008
}

 
%/////////////////////////////////////////////////////////////
 
\mbox{}\hrulefill\ 
 
\subsection{getngqmax (Source File: getngqmax.F90)}


\bigskip{\sf INTERFACE:}
\begin{verbatim} subroutine getngqmax\end{verbatim}{\em USES:}
\begin{verbatim} use modmain
 use modxs\end{verbatim}
{\sf DESCRIPTION:\\ }


     Determines the largest number of ${\bf G+q}$-vectors with length less than
     {\tt gqmax} over all the ${\bf q}$-points and stores it in the global
     variable {\tt ngqmax}. This variable is used for allocating arrays.
     Based upon the routine getngkmax.
  
\bigskip{\sf REVISION HISTORY:}
\begin{verbatim}     Created October 2006 (Sagmeister)\end{verbatim}

\markboth{Left}{Source File: getoccsv0.F90,  Date: Fri May 23 14:38:01 CEST 2008
}


\markboth{Left}{Source File: getpemat.F90,  Date: Fri May 23 14:38:01 CEST 2008
}


\markboth{Left}{Source File: getpmat.F90,  Date: Fri May 23 14:38:01 CEST 2008
}


\markboth{Left}{Source File: getridx.F90,  Date: Fri May 23 14:38:01 CEST 2008
}


\markboth{Left}{Source File: gettetcw.F90,  Date: Fri May 23 14:38:01 CEST 2008
}


\markboth{Left}{Source File: getunit.F90,  Date: Fri May 23 14:38:01 CEST 2008
}


\markboth{Left}{Source File: getx0.F90,  Date: Fri May 23 14:38:01 CEST 2008
}


\markboth{Left}{Source File: gndstateq.F90,  Date: Fri May 23 14:38:01 CEST 2008
}


\markboth{Left}{Source File: gradzfmtr.F90,  Date: Fri May 23 14:38:01 CEST 2008
}

 
%/////////////////////////////////////////////////////////////
 
\mbox{}\hrulefill\ 
 
\subsection{gradzfmtr (Source File: gradzfmtr.F90)}


\bigskip{\sf INTERFACE:}
\begin{verbatim} subroutine gradzfmtr(lmax,nr,r,l1,m1,ld1,ld2,fmt,gfmt)\end{verbatim}{\em USES:}
\begin{verbatim}   use modmain, only: idxlm\end{verbatim}{\em INPUT/OUTPUT PARAMETERS:}
\begin{verbatim}     lmax  : maximum angular momentum (in,integer)
     nr    : number of radial mesh points (in,integer)
     r     : radial mesh (in,real(nr))
     ld1   : leading dimension 1 (in,integer)
     ld2   : leading dimension 2 (in,integer)
     fmt  : real muffin-tin function (in,real(nr))
     gfmt : gradient of zfmt (out,real(ld1,ld2,3))\end{verbatim}
{\sf DESCRIPTION:\\ }


     Calculates the gradient of a muffin-tin function with real spherical 
     harmonics expansion coefficients, $f(r)$, corresponding to a specific
     $lm$-combination. The gradient is given in a spherical harmonics
     representation.
     The $y$-component is divided by $i$ to be expressed as a real number.
     See routine {\tt gradzfmt}.
  
\bigskip{\sf REVISION HISTORY:}
\begin{verbatim}     Created April 2008 (Sagmeister)\end{verbatim}

\markboth{Left}{Source File: i2str.F90,  Date: Fri May 23 14:38:01 CEST 2008
}


\markboth{Left}{Source File: idf.F90,  Date: Fri May 23 14:38:01 CEST 2008
}


\markboth{Left}{Source File: idfgather.F90,  Date: Fri May 23 14:38:01 CEST 2008
}


\markboth{Left}{Source File: idfq.F90,  Date: Fri May 23 14:38:01 CEST 2008
}


\markboth{Left}{Source File: init1xs.F90,  Date: Fri May 23 14:38:01 CEST 2008
}


\markboth{Left}{Source File: init2xs.F90,  Date: Fri May 23 14:38:01 CEST 2008
}


\markboth{Left}{Source File: initbse.F90,  Date: Fri May 23 14:38:01 CEST 2008
}


\markboth{Left}{Source File: initoccbse.F90,  Date: Fri May 23 14:38:01 CEST 2008
}


\markboth{Left}{Source File: initscr.F90,  Date: Fri May 23 14:38:01 CEST 2008
}


\markboth{Left}{Source File: invert.F90,  Date: Fri May 23 14:38:01 CEST 2008
}


\markboth{Left}{Source File: kernxc\_bse.F90,  Date: Fri May 23 14:38:01 CEST 2008
}


\markboth{Left}{Source File: kernxc.F90,  Date: Fri May 23 14:38:01 CEST 2008
}

 
%/////////////////////////////////////////////////////////////
 
\mbox{}\hrulefill\ 
 
\subsection{fxc\_alda (Source File: kernxc.F90)}


\bigskip{\sf INTERFACE:}
\begin{verbatim} subroutine kernxc\end{verbatim}
{\sf DESCRIPTION:\\ }


     Computes the ALDA exchange-correlation kernel. In the
     muffin-tin, the density is transformed from spherical harmonic coefficients
     $\rho_{lm}$ to spherical coordinates $(\theta,\phi)$ with a backward
     spherical harmonic transformation (SHT). Once calculated, the
     exchange-correlation potential and energy density are transformed with a
     forward SHT. This routine is based upon the routine {\tt potxc}.
  
\bigskip{\sf REVISION HISTORY:}
\begin{verbatim}     Created March 2007 (Sagmeister)\end{verbatim}

\markboth{Left}{Source File: kramkron.F90,  Date: Fri May 23 14:38:01 CEST 2008
}


\markboth{Left}{Source File: l2int.F90,  Date: Fri May 23 14:38:01 CEST 2008
}


\markboth{Left}{Source File: linoptkpq.F90,  Date: Fri May 23 14:38:01 CEST 2008
}


\markboth{Left}{Source File: linoptold.F90,  Date: Fri May 23 14:38:01 CEST 2008
}


\markboth{Left}{Source File: mapto1bz.F90,  Date: Fri May 23 14:38:01 CEST 2008
}


\markboth{Left}{Source File: modfxcifc.F90,  Date: Fri May 23 14:38:01 CEST 2008
}

 
%/////////////////////////////////////////////////////////////
 
\mbox{}\hrulefill\ 
 
\subsection{fxcifc (Source File: modfxcifc.F90)}


\bigskip{\sf INTERFACE:}
\begin{verbatim}   subroutine fxcifc(fxctype,w,iw,iq,ng,alrc,alrcd,blrcd,lrccoef,fxcg)
     use m_fxc_lrc
     use m_fxc_lrcd
      use m_fxc_lrcmodel
     use m_fxc_alda
     use m_fxc_bse_ma03\end{verbatim}{\em INPUT/OUTPUT PARAMETERS:}
\begin{verbatim}     fxctype : type of exchange-correlation functional (in,integer)\end{verbatim}
{\sf DESCRIPTION:\\ }


     Interface to the exchange-correlation kernel routines. This makes it 
     relatively
     simple to add new functionals which do not necessarily depend only on
     all input parameters. Based upon the routine {\tt modxcifc}.
  
\bigskip{\sf REVISION HISTORY:}
\begin{verbatim}     Created October 2007 (Sagmeister)\end{verbatim}
 
%/////////////////////////////////////////////////////////////
 
\mbox{}\hrulefill\ 
 
\subsection{getfxcdata (Source File: modfxcifc.F90)}


\bigskip{\sf INTERFACE:}
\begin{verbatim}   subroutine getfxcdata(fxctype,fxcdescr,fxcspin)\end{verbatim}{\em INPUT/OUTPUT PARAMETERS:}
\begin{verbatim}     fxctype  : type of exchange-correlation functional (in,integer)
     fxcdescr : description of functional (out,character(256))
     fxcspin  : spin treatment (out,integer)
     fxcgrad  : gradient treatment (out,integer)\end{verbatim}
{\sf DESCRIPTION:\\ }


     Returns data on the exchange-correlation functional labelled by
     {\tt fxctype}. The character array {\tt fxctype} contains a short
     description
     of the functional including journal references. The variable 
     {\tt fxcspin} is
     set to 1 or 0 for spin-polarised or -unpolarised functionals,
     respectively.
  
\bigskip{\sf REVISION HISTORY:}
\begin{verbatim}     Created October 2007 (Sagmeister)\end{verbatim}

\markboth{Left}{Source File: modifcs.F90,  Date: Fri May 23 14:38:01 CEST 2008
}


\markboth{Left}{Source File: modxs.F90,  Date: Fri May 23 14:38:01 CEST 2008
}


\markboth{Left}{Source File: octmap.F90,  Date: Fri May 23 14:38:01 CEST 2008
}


\markboth{Left}{Source File: pade.F90,  Date: Fri May 23 14:38:01 CEST 2008
}


\markboth{Left}{Source File: pmatgather.F90,  Date: Fri May 23 14:38:01 CEST 2008
}


\markboth{Left}{Source File: pmatrad.F90,  Date: Fri May 23 14:38:01 CEST 2008
}


\markboth{Left}{Source File: pmattd2orig.F90,  Date: Fri May 23 14:38:01 CEST 2008
}


\markboth{Left}{Source File: putemat.F90,  Date: Fri May 23 14:38:01 CEST 2008
}


\markboth{Left}{Source File: putpmat.F90,  Date: Fri May 23 14:38:01 CEST 2008
}


\markboth{Left}{Source File: puttetcw.F90,  Date: Fri May 23 14:38:01 CEST 2008
}


\markboth{Left}{Source File: putx0.F90,  Date: Fri May 23 14:38:01 CEST 2008
}


\markboth{Left}{Source File: r2str.F90,  Date: Fri May 23 14:38:01 CEST 2008
}


\markboth{Left}{Source File: scrcoulint2.F90,  Date: Fri May 23 14:38:01 CEST 2008
}


\markboth{Left}{Source File: scrcoulint3.F90,  Date: Fri May 23 14:38:01 CEST 2008
}


\markboth{Left}{Source File: scrcoulint.F90,  Date: Fri May 23 14:38:01 CEST 2008
}


\markboth{Left}{Source File: screen.F90,  Date: Fri May 23 14:38:01 CEST 2008
}


\markboth{Left}{Source File: scrgeneigvec.F90,  Date: Fri May 23 14:38:01 CEST 2008
}


\markboth{Left}{Source File: scrtetcalccw.F90,  Date: Fri May 23 14:38:01 CEST 2008
}


\markboth{Left}{Source File: showunits.F90,  Date: Fri May 23 14:38:01 CEST 2008
}


\markboth{Left}{Source File: sleepifc.F90,  Date: Fri May 23 14:38:01 CEST 2008
}


\markboth{Left}{Source File: stringtim.F90,  Date: Fri May 23 14:38:01 CEST 2008
}


\markboth{Left}{Source File: summations.F90,  Date: Fri May 23 14:38:01 CEST 2008
}


\markboth{Left}{Source File: symg2f.F90,  Date: Fri May 23 14:38:01 CEST 2008
}

 
%/////////////////////////////////////////////////////////////
 
\mbox{}\hrulefill\ 
 
\subsection{symg2f (Source File: symg2f.F90)}


\bigskip{\sf INTERFACE:}
\begin{verbatim} subroutine symg2f(vpl,ngp,igpig,fg)\end{verbatim}{\em USES:}
\begin{verbatim} use modmain
 use modxs
 \end{verbatim}{\em INPUT/OUTPUT PARAMETERS:}
\begin{verbatim}     vpl     :vpl
     ngp     :ngp
     igpig   :igpig
     fg      :fg\end{verbatim}
{\sf DESCRIPTION:\\ }


     Symmetrises a real scalar function given in $G$-space
     $$ f_{\bf q}({\bf G},{\bf G'}) = \frac{1}{\Omega}\int {\rm d}^3r 
     e^{-i{\bf (G+q)r}} f({\bf r},{\bf r'}) e^{i{\bf (G'}+{\bf q}){\bf r'}} $$
     of which its real-space representation is invariant under application of
     a symmetry  operation to both spatial variables, i.e.:
     $$ f({\bf r},{\bf r'}) = f(\{\alpha|\tau\}{\bf r},\{\alpha|\tau\}{\bf r'}).
     $$
     The symmetry operations are restricted to the subset that leaves the
     ${\bf q}$-vector unaltered
     $$ {\bf q}=\alpha^{-1}{\bf q} + {\bf G}_{\alpha},  $$
     building up the small (little) group of {\bf q}.
     For a function obeying the latter symmetry property one can derive a
     symmetry relation for the Fourier-coefficients as well:
     $$ f_{\bf q}({\bf G},{\bf G'}) =  
     e^{i({\bf G'}-{\bf G})\tau_{\alpha^{-1}}} 
     f_{\bf q}(\alpha[{\bf G}+{\bf G}_{\alpha}],
     \alpha[{\bf G'}+{\bf G}_{\alpha}]) =: f^{(\alpha)}_{\bf q}({\bf G},
     {\bf G'}).$$
     Since this property may not exactly be fulfilled numerically, an average
     of the rotated expressions $f^{(\alpha)}$ is taken:
     $$  \bar{f}_{\bf q}({\bf G},{\bf G'})= \frac{1}{N_{\bf q}}
     \sum_{\alpha}^{\mathcal{G}({\bf q})}
     f^{(\alpha)}_{\bf q}({\bf G},{\bf G'}) $$
     where $\mathcal{G}({\bf q})$ denotes the small (little) group of ${\bf q}$
     and $N_{\bf q}$ its number of elements.
  
\bigskip{\sf REVISION HISTORY:}
\begin{verbatim}     Created March 2008 (SAG)\end{verbatim}

\markboth{Left}{Source File: symsci0.F90,  Date: Fri May 23 14:38:01 CEST 2008
}


\markboth{Left}{Source File: tdchkstop.F90,  Date: Fri May 23 14:38:01 CEST 2008
}


\markboth{Left}{Source File: tdgauntgen.F90,  Date: Fri May 23 14:38:01 CEST 2008
}


\markboth{Left}{Source File: tdlinopt.F90,  Date: Fri May 23 14:38:01 CEST 2008
}


\markboth{Left}{Source File: tdsave0.F90,  Date: Fri May 23 14:38:01 CEST 2008
}


\markboth{Left}{Source File: tdwriteh.F90,  Date: Fri May 23 14:38:01 CEST 2008
}


\markboth{Left}{Source File: tdzoutpr2.F90,  Date: Fri May 23 14:38:01 CEST 2008
}


\markboth{Left}{Source File: tdzoutpr.F90,  Date: Fri May 23 14:38:01 CEST 2008
}


\markboth{Left}{Source File: terminate.F90,  Date: Fri May 23 14:38:01 CEST 2008
}


\markboth{Left}{Source File: testmain.F90,  Date: Fri May 23 14:38:01 CEST 2008
}


\markboth{Left}{Source File: testxs.F90,  Date: Fri May 23 14:38:01 CEST 2008
}


\markboth{Left}{Source File: tetcalccw.F90,  Date: Fri May 23 14:38:01 CEST 2008
}


\markboth{Left}{Source File: tetcalccwq.F90,  Date: Fri May 23 14:38:01 CEST 2008
}


\markboth{Left}{Source File: tetgather.F90,  Date: Fri May 23 14:38:01 CEST 2008
}


\markboth{Left}{Source File: tqgamma.F90,  Date: Fri May 23 14:38:01 CEST 2008
}


\markboth{Left}{Source File: updateq.F90,  Date: Fri May 23 14:38:01 CEST 2008
}


\markboth{Left}{Source File: wavefmt\_apw.F90,  Date: Fri May 23 14:38:01 CEST 2008
}

 
%/////////////////////////////////////////////////////////////
 
\mbox{}\hrulefill\ 
 
\subsection{wavefmt\_apw (Source File: wavefmt\_apw.F90)}


\bigskip{\sf INTERFACE:}
\begin{verbatim} subroutine wavefmt_apw(lrstp,lmax,is,ia,ngp,apwalm,evecfv,ld,wfmt)\end{verbatim}{\em USES:}
\begin{verbatim} use modmain\end{verbatim}{\em INPUT/OUTPUT PARAMETERS:}
\begin{verbatim}     lrstp  : radial step length (in,integer)
     lmax   : maximum angular momentum required (in,integer)
     is     : species number (in,integer)
     ia     : atom number (in,integer)
     ngp    : number of G+p-vectors (in,integer)
     apwalm : APW matching coefficients
              (in,complex(ngkmax,apwordmax,lmmaxapw,natmtot))
     evecfv : first-variational eigenvector (in,complex(nmatmax))
     ld     : leading dimension (in,integer)
     wfmt   : muffin-tin wavefunction (out,complex(ld,*))\end{verbatim}
{\sf DESCRIPTION:\\ }


  
\bigskip{\sf REVISION HISTORY:}
\begin{verbatim}     Created April 2003 (JKD)
     Fixed description, October 2004 (C. Brouder)
     Removed argument ist, November 2006 (JKD)\end{verbatim}

\markboth{Left}{Source File: wavefmt\_lo.F90,  Date: Fri May 23 14:38:01 CEST 2008
}

 
%/////////////////////////////////////////////////////////////
 
\mbox{}\hrulefill\ 
 
\subsection{wavefmt\_lo (Source File: wavefmt\_lo.F90)}


\bigskip{\sf INTERFACE:}
\begin{verbatim} subroutine wavefmt_lo(lrstp,lmax,is,ia,ngp,apwalm,evecfv,ld,wfmt)\end{verbatim}{\em USES:}
\begin{verbatim} use modmain\end{verbatim}{\em INPUT/OUTPUT PARAMETERS:}
\begin{verbatim}     lrstp  : radial step length (in,integer)
     lmax   : maximum angular momentum required (in,integer)
     is     : species number (in,integer)
     ia     : atom number (in,integer)
     ngp    : number of G+p-vectors (in,integer)
     apwalm : APW matching coefficients
              (in,complex(ngkmax,apwordmax,lmmaxapw,natmtot))
     evecfv : first-variational eigenvector (in,complex(nmatmax))
     ld     : leading dimension (in,integer)
     wfmt   : muffin-tin wavefunction (out,complex(ld,*))\end{verbatim}
{\sf DESCRIPTION:\\ }


  
\bigskip{\sf REVISION HISTORY:}
\begin{verbatim}     Created April 2003 (JKD)
     Fixed description, October 2004 (C. Brouder)
     Removed argument ist, November 2006 (JKD)\end{verbatim}

\markboth{Left}{Source File: writeangmom.F90,  Date: Fri May 23 14:38:01 CEST 2008
}

 
%/////////////////////////////////////////////////////////////
 
\mbox{}\hrulefill\ 
 
\subsection{writeangmom (Source File: writeangmom.F90)}


\bigskip{\sf INTERFACE:}
\begin{verbatim} subroutine writeangmom(un)\end{verbatim}{\em USES:}
\begin{verbatim}   use modmain
   use modxs\end{verbatim}{\em INPUT/OUTPUT PARAMETERS:}
\begin{verbatim} \end{verbatim}
{\sf DESCRIPTION:\\ }

    Outputs information about the angular momentum cutoffs to the file
     {\tt INFO\_XS.OUT}.
  
\bigskip{\sf REVISION HISTORY:}
\begin{verbatim}     Created October 2006 (Sagmeister)\end{verbatim}

\markboth{Left}{Source File: writeemat\_ascii.F90,  Date: Fri May 23 14:38:01 CEST 2008
}


\markboth{Left}{Source File: writeemat.F90,  Date: Fri May 23 14:38:01 CEST 2008
}


\markboth{Left}{Source File: writeeps.F90,  Date: Fri May 23 14:38:01 CEST 2008
}


\markboth{Left}{Source File: writeexciton.F90,  Date: Fri May 23 14:38:01 CEST 2008
}


\markboth{Left}{Source File: writegqpts.F90,  Date: Fri May 23 14:38:01 CEST 2008
}

 
%/////////////////////////////////////////////////////////////
 
\mbox{}\hrulefill\ 
 
\subsection{writegqpts (Source File: writegqpts.F90)}


\bigskip{\sf INTERFACE:}
\begin{verbatim}   subroutine writegqpts(iq)\end{verbatim}{\em USES:}
\begin{verbatim}     use modmain
     use modxs
     use m_getunit\end{verbatim}
{\sf DESCRIPTION:\\ }


     Writes the ${\bf G+q}$-points in lattice coordinates, Cartesian 
     coordinates, and lengths of ${\bf G+q}$-vectors to the file 
     {\tt QPOINTS.OUT}. Based on the routine {\tt writekpts.f90}.
  
\bigskip{\sf REVISION HISTORY:}
\begin{verbatim}     Created October 2006 (Sagmeister)\end{verbatim}

\markboth{Left}{Source File: writekmapkq.F90,  Date: Fri May 23 14:38:01 CEST 2008
}


\markboth{Left}{Source File: writeloss.F90,  Date: Fri May 23 14:38:01 CEST 2008
}


\markboth{Left}{Source File: writepmat\_ascii.F90,  Date: Fri May 23 14:38:01 CEST 2008
}


\markboth{Left}{Source File: writepmatxs.F90,  Date: Fri May 23 14:38:01 CEST 2008
}

 
%/////////////////////////////////////////////////////////////
 
\mbox{}\hrulefill\ 
 
\subsection{writepmatxs (Source File: writepmatxs.F90)}


\bigskip{\sf INTERFACE:}
\begin{verbatim} subroutine writepmatxs(lgather)\end{verbatim}{\em USES:}
\begin{verbatim}   use modmain
   use modmpi
   use modxs
   use m_putpmat
   use m_genfilname\end{verbatim}
{\sf DESCRIPTION:\\ }


     Calculates the momentum matrix elements using routine {\tt genpmat} and
     writes them to direct access file {\tt PMAT\_XS.OUT}. Derived from
     the routine {\tt writepmat}.
  
\bigskip{\sf REVISION HISTORY:}
\begin{verbatim}     Created 2006 (Sagmeister)\end{verbatim}

\markboth{Left}{Source File: writepwmat.F90,  Date: Fri May 23 14:38:01 CEST 2008
}

 
%/////////////////////////////////////////////////////////////
 
\mbox{}\hrulefill\ 
 
\subsection{writepwmat (Source File: writepwmat.F90)}


\bigskip{\sf INTERFACE:}
\begin{verbatim} subroutine writepwmat\end{verbatim}{\em USES:}
\begin{verbatim}   use modmain
   use modxs
   use m_genfilname\end{verbatim}
{\sf DESCRIPTION:\\ }


     Calculates the matrix elements of the plane wave
     $e^{-i({\bf G}+{\bf q}){\bf r}}$
     using routine {\tt genpwmat} and writes them to
     direct access file {\tt PWMAT.OUT}.
  
\bigskip{\sf REVISION HISTORY:}
\begin{verbatim}     Created November 2007 (Sagmeister)\end{verbatim}

\markboth{Left}{Source File: writeqpts.F90,  Date: Fri May 23 14:38:01 CEST 2008
}

 
%/////////////////////////////////////////////////////////////
 
\mbox{}\hrulefill\ 
 
\subsection{writeqpts (Source File: writeqpts.F90)}


\bigskip{\sf INTERFACE:}
\begin{verbatim} subroutine writeqpts\end{verbatim}{\em USES:}
\begin{verbatim} use modmain
 use modxs
 use m_getunit
 use m_genfilname\end{verbatim}
{\sf DESCRIPTION:\\ }


     Writes the ${\bf q}$-points in lattice coordinates, weights and number of
     ${\bf G+q}$-vectors to the file {\tt QPOINTS.OUT}. Based on the routine 
     {\tt writekpts}.
  
\bigskip{\sf REVISION HISTORY:}
\begin{verbatim}     Created October 2006 (Sagmeister)\end{verbatim}

\markboth{Left}{Source File: writesigma.F90,  Date: Fri May 23 14:38:01 CEST 2008
}


\markboth{Left}{Source File: writesumrls.F90,  Date: Fri May 23 14:38:01 CEST 2008
}


\markboth{Left}{Source File: writesymi.F90,  Date: Fri May 23 14:38:01 CEST 2008
}

 
%/////////////////////////////////////////////////////////////
 
\mbox{}\hrulefill\ 
 
\subsection{writesymi (Source File: writesymi.F90)}


\bigskip{\sf INTERFACE:}
\begin{verbatim} subroutine writesymi\end{verbatim}{\em USES:}
\begin{verbatim}   use modmain
   use modxs\end{verbatim}
{\sf DESCRIPTION:\\ }


     Outputs the crystal and symmetries including their inverse
     elements to file {\tt SYMINV.OUT}
  
\bigskip{\sf REVISION HISTORY:}
\begin{verbatim}     Created December 2007 (Sagmeister)\end{verbatim}

\markboth{Left}{Source File: x0toasc.F90,  Date: Fri May 23 14:38:01 CEST 2008
}


\markboth{Left}{Source File: x0tobin.F90,  Date: Fri May 23 14:38:01 CEST 2008
}


\markboth{Left}{Source File: xscheck.F90,  Date: Fri May 23 14:38:01 CEST 2008
}


\markboth{Left}{Source File: xsestimate.F90,  Date: Fri May 23 14:38:01 CEST 2008
}


\markboth{Left}{Source File: xsfinit.F90,  Date: Fri May 23 14:38:01 CEST 2008
}


\markboth{Left}{Source File: xsgeneigvec.F90,  Date: Fri May 23 14:38:01 CEST 2008
}


\markboth{Left}{Source File: xsinit.F90,  Date: Fri May 23 14:38:01 CEST 2008
}


\markboth{Left}{Source File: zaxpyc.F90,  Date: Fri May 23 14:38:01 CEST 2008
}


\markboth{Left}{Source File: zfinp2.F90,  Date: Fri May 23 14:38:01 CEST 2008
}

 
%/////////////////////////////////////////////////////////////
 
\mbox{}\hrulefill\ 
 
\subsection{zfinp2 (Source File: zfinp2.F90)}


\bigskip{\sf INTERFACE:}
\begin{verbatim} complex(8) function zfinp2(ngp1,ngp2,igpig,zfmt1,zfmt2,zfir1,zfir2)\end{verbatim}{\em USES:}
\begin{verbatim}   use modmain\end{verbatim}{\em INPUT/OUTPUT PARAMETERS:}
\begin{verbatim}     zfmt1 : first complex function in spherical harmonics for all muffin-tins
             (in,complex(lmmaxvr,nrcmtmax,natmtot))
     zfmt2 : second complex function in spherical harmonics for all muffin-tins
             (in,complex(lmmaxvr,nrcmtmax,natmtot))
     zfir1 : first complex interstitial function in real-space
             (in,complex(ngrtot))
     zfir2 : second complex interstitial function in real-space
             (in,complex(ngrtot))\end{verbatim}
{\sf DESCRIPTION:\\ }


     Calculates the inner product of two complex fuctions over the entire unit
     cell. The muffin-tin functions should be stored on the coarse radial grid
     and have angular momentum cut-off {\tt lmaxvr}. In the intersitial region,
     the integrand is multiplied with the smooth characteristic function,
     $\tilde{\Theta}({\bf r})$, to remove the contribution from the muffin-tin.
     See routines {\tt zfmtinp} and {\tt gencfun}.
  
\bigskip{\sf REVISION HISTORY:}
\begin{verbatim}     Created July 2004 (Sharma)
     Modifications Jannuary 2007 (Sagmeister)\end{verbatim}

\markboth{Left}{Source File: zoutpr.F90,  Date: Fri May 23 14:38:01 CEST 2008
}

 
%/////////////////////////////////////////////////////////////
 
\mbox{}\hrulefill\ 
 
\subsection{zoutpr (Source File: zoutpr.F90)}


\bigskip{\sf INTERFACE:}
\begin{verbatim} subroutine zoutpr(n1,n2,alpha,x,y,a)\end{verbatim}{\em INPUT/OUTPUT PARAMETERS:}
\begin{verbatim}     n1,n2 : size of vectors and matrix, respectively (in,integer)
     alpha : complex constant (in,complex)
     x     : first input vector (in,complex(n1))
     y     : second input vector (in,complex(n2))
     a     : output matrix (out,complex(n1,n2))\end{verbatim}
{\sf DESCRIPTION:\\ }


     Performs the rank-2 operation
     $$ A_{ij}\rightarrow\alpha{\bf x}_i^*{\bf y}_j+A_{ij}. $$
  
\bigskip{\sf REVISION HISTORY:}
\begin{verbatim}     Created April 2008 (Sagmeister)\end{verbatim}

%...............................................................
\end{document}
